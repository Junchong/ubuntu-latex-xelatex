\documentclass[12pt,a4paper]{article}
\usepackage{xeCJK}
\usepackage{fontspec,xunicode,xltxtra}
\usepackage{titlesec}
\usepackage[colorlinks,linkcolor=red]{hyperref}

\setmainfont{Times New Roman}%缺省英文字体.serif是有衬线字体sans serif无衬线字体
\setCJKmainfont{SimSun}

\setCJKfamilyfont{song}{SimSun}                             %宋体 song
\newcommand{\song}{\CJKfamily{song}}
\setCJKfamilyfont{yahei}{Microsoft YaHei}                    %微软雅黑 yh
\newcommand{\yahei}{\CJKfamily{yahei}}
\setCJKfamilyfont{fs}{FangSong}                    %仿宋
\newcommand{\fs}{\CJKfamily{fs}}
\setCJKfamilyfont{kt}{KaiTi}                    %楷体
\newcommand{\kt}{\CJKfamily{kt}}

\begin{document}
\title{ubuntu Atom \LaTeX (xelatex) \yahei{中文字体测试}}
\author{\kt{马俊冲}}
\date{\fs{2017年8月}}
\maketitle

\section{\yahei{环境搭建}}
\subsection{Texlive安装}
Texlive2017.iso 下载地址:\url{http://mirrors.ustc.edu.cn/CTAN/systems/texlive/Images/}
安装和配置可参考网上博客,注意配置版本文件。
具体安装细节,可以参考\href{http://www.linuxidc.com/Linux/2016-08/133913.htm}
{Linux 系统下原版 TeX Live 2016 的安装与配置}。\\
此外,为了使用图形化安装界面,需要\yahei{安装perl的tk组件:\\
${sudo\ apt-get\ install\ perl-tk}$\\
加载镜像文件:\\
${sudo\ mount\ -o\ loop\ texlive2016.iso\ /mnt}$\\
启动安装程序的图形化界面进行配置:\\
${cd\ /mnt}$\\
${sudo\ ./install-tl\ -gui}$\\
完成后,卸载镜像文件:}\\
${cd\ /;\ sudo\ umount\ /mnt}$\\
进一步需要对字体配置,配置自动更新源,配置安装dummy package包等,查看上述链接。

\subsection{安装xelatex}
命令:\\
${sudo\ apt-get\ install\ texlive-xetex}$
关于CJK字体的安装,可以参考网上博客
\url{http://www.jianshu.com/p/d185aad1f915}\\
或者\\
\url{http://www.cnblogs.com/bamboo-talking/archive/2013/01/07/2848914.html}\\

\section{\yahei{Atom安装和配置}}
\subsection{\yahei{Atom编辑器安装}}
下载安装Atom编辑器,从官网\url{https://atom.io/}下载(网速略慢)deb安装。\\
${sudo\ dpkg\ -i\ xxx.deb}$\\
\subsubsection{\yahei{Atom插件安装}}
为实现 \LaTeX 文档的编译、显示、自动补全等搜索并安装以下3个插件:\\
latex\\
language-latex\\
pdf-view\\
\subsection{\yahei{插件latex的配置}}
Tex Path:\\
${/usr/local/bin/latexmk}$\\
勾选${open\ result\ after\ successful\ build}$\\
选择Opener为pdf-view\\
编译快捷键为ctrl+alt+b\\

\section{\yahei{文档中基于xelatex的字体配置}}
可以参考以下配置:
\begin{verbatim}
  \documentclass[12pt,a4paper]{article}
  \usepackage{xeCJK}
  \usepackage{fontspec,xunicode,xltxtra}
  \usepackage{titlesec}
  \usepackage[colorlinks,linkcolor=red]{hyperref}

  \setmainfont{Times New Roman}%缺省英文字体.serif是有衬线字体sans serif无衬线字体
  \setCJKmainfont{SimSun}

  \setCJKfamilyfont{song}{SimSun}                             %宋体 song
  \newcommand{\song}{\CJKfamily{song}}
  \setCJKfamilyfont{yahei}{Microsoft YaHei}                    %微软雅黑 yh
  \newcommand{\yahei}{\CJKfamily{yahei}}
  \setCJKfamilyfont{fs}{FangSong}                    %仿宋
  \newcommand{\fs}{\CJKfamily{fs}}
  \setCJKfamilyfont{kt}{KaiTi}                    %楷体
  \newcommand{\kt}{\CJKfamily{kt}}

  \begin{document}
  \title{ubuntu Atom \LaTeX (xelatex) \yahei{中文字体测试}}
  \author{\kt{马俊冲}}
  \date{\fs{2017年8月}}
  \maketitle

\end{verbatim}



I just 想 测试 \song{宋体} \fs{仿宋} \yahei{微软雅黑} \kt{楷体}\\
\\
And Happy With \LaTeX
\end{document}
